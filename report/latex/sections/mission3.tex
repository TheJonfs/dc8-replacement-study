\subsection{Mission Definition}

\begin{table}[htbp]
  \centering
  \caption[Mission 3 parameters]{Mission 3 definition parameters.}
  \label{tab:m3-definition}
  \begin{tabular}{ll}
    \toprule
    Parameter & Value \\
    \midrule
    Region & Central United States (Arkansas-Missouri area) \\
    Duration & 8 hours \\
    Payload & 30,000 lb \\
    Altitude & 1,500 ft above ground level \\
    Speed & 250 KTAS (Mach 0.38) \\
    Fuel budget & Explicit reserves + iterative mission-sized loading \\
    \bottomrule
  \end{tabular}
\end{table}

This mission represents extended low-altitude operations for forest fire particulate sampling. Unlike Missions~1 and~2, the constraint is time rather than distance: the aircraft must sustain 8 hours of flight at low altitude with the specified payload. The slow speed (250~KTAS, well below cruise Mach for all candidates) and low altitude (1,500~ft, far below optimal cruise altitude) represent severely off-design operating conditions.

\subsection{Mission-Sized Fuel Loading}

A critical modeling decision for Mission~3 is fuel loading. If aircraft carry maximum fuel, the large-tank aircraft (777-200LR with 325,300~lb capacity, A330-200 with 245,264~lb) are heavily penalized: the extra fuel weight increases drag and fuel consumption, creating a self-reinforcing weight spiral.

The solution is iterative mission-sized fuel loading. Each aircraft loads only enough fuel for the 8-hour mission plus reserves:

\begin{table}[htbp]
  \centering
  \caption[Mission 3 fuel loading]{Mission-sized fuel loading for Mission~3. Each aircraft loads only enough fuel for the 8-hour endurance requirement plus reserves.}
  \label{tab:m3-fuel-loading}
  \begin{tabular}{lrrr}
    \toprule
    Aircraft & Max Fuel (lb) & Mission Fuel Loaded (lb) & Utilization \\
    \midrule
    DC-8 & 147,255 & 44,496 & 30\% \\
    GV & 41,300 & 23,393 & 57\% \\
    P-8 & 73,320 & 37,321 & 51\% \\
    767-200ER & 162,000 & 96,124 & 59\% \\
    A330-200 & 245,264 & 166,342 & 68\% \\
    777-200LR & 325,300 & 128,747 & 40\% \\
    \bottomrule
  \end{tabular}
\end{table}

All aircraft converged within 5--8 iterations of the fuel sizing algorithm (tolerance: 50~lb). The impact is dramatic for the 777-200LR: without mission-sized loading, its cost metric would be \$4.45/klb-nm; with it, the cost drops to \$1.76/klb-nm---a 60\% reduction.

\subsection{Results}

\subsubsection{Feasibility Assessment}

All six aircraft pass Mission~3. This is the least discriminating mission: the 30,000~lb payload and 8-hour endurance requirement are within the capability of all candidates.

\begin{table}[htbp]
  \centering
  \caption[Mission 3 feasibility]{Mission 3 feasibility assessment. All aircraft complete the 8-hour endurance requirement.}
  \label{tab:m3-feasibility}
  \small
  \begin{tabular}{lccrrr}
    \toprule
    Aircraft & Confidence & $n$ & Fuel Burned (lb) & Avg Flow (lb/hr) & Endurance (hr) \\
    \midrule
    DC-8* & Low & 1 & 36,233 & 4,529 & 8.0 \\
    GV & Medium & 6 & 18,938 ea & 2,367 ea & 8.0 \\
    P-8 & Low & 2 & 30,244 ea & 3,780 ea & 8.0 \\
    \textbf{767-200ER} & \textbf{High} & 1 & 78,921 & 9,865 & 8.0 \\
    A330-200 & Low & 1 & 133,649 & 16,706 & 8.0 \\
    777-200LR & Low & 1 & 102,996 & 12,874 & 8.0 \\
    \bottomrule
  \end{tabular}

  \smallskip
  {\footnotesize *DC-8 burn rate approximately 40\% below reality due to $k_{\text{adj}} = 0.605$.}
\end{table}

\subsubsection{Weight Breakdown}

\cref{fig:weight-breakdown-m3} shows the weight breakdown for Mission~3.

\begin{figure}[htbp]
  \centering
  \includegraphics[width=0.85\textwidth]{figures/weight_breakdown_m3.png}
  \caption[Mission 3 weight breakdown]{Weight breakdown comparison for Mission~3 (low-altitude smoke survey). Mission-sized fuel loading is visible as smaller fuel fractions relative to OEW compared with Missions~1 and~2. Fleet aggregates are shown for the GV~($6\times$) and P-8~($2\times$).}
  \label{fig:weight-breakdown-m3}
\end{figure}

The mission-sized fuel loading is immediately visible: fuel fractions are much smaller relative to OEW than in Missions~1--2. The DC-8's fuel band (44,500~lb) is notably thin compared to its 147,255~lb capacity. The GV $6\times$ fleet aggregate (460,000~lb) remains larger than the single 767 (305,000~lb) and A330 (462,000~lb), reinforcing the fleet weight penalty.

\subsubsection{Low-Altitude Aerodynamics}

Operating at 1,500~ft and Mach~0.38 places all aircraft far below their design cruise conditions. At these conditions, parasite drag ($C_{D_0}$) dominates:

\begin{table}[htbp]
  \centering
  \caption[Mission 3 aerodynamics]{Low-altitude aerodynamic parameters for Mission~3. Aircraft with unphysical $C_{D_0}$ calibrations show parasite drag fractions exceeding 84\%.}
  \label{tab:m3-aero}
  \begin{tabular}{lcccc}
    \toprule
    Aircraft & $C_L$ & $C_D$ & $L/D$ & $C_{D_0}/C_D$ \\
    \midrule
    DC-8 & 0.399 & 0.023 & 17.6 & 62\% \\
    GV & 0.333 & 0.021 & 15.8 & 71\% \\
    P-8 & 0.527 & 0.071 & 7.4 & 84\% \\
    767-200ER & 0.494 & 0.032 & 15.6 & 56\% \\
    A330-200 & 0.587 & 0.038 & 15.3 & 87\% \\
    777-200LR & 0.503 & 0.047 & 10.6 & 87\% \\
    \bottomrule
  \end{tabular}
\end{table}

For aircraft with unphysical $C_{D_0}$ (P-8, A330, 777), the parasite drag fraction exceeds 84\%, meaning their fuel burn is dominated by the calibration artifact rather than physics. The DC-8 and 767 have more balanced drag polars, with the 767 showing the lowest parasite drag fraction (56\%) and a healthy $L/D$ of 15.6.

\subsection{Fuel Cost}

\begin{table}[htbp]
  \centering
  \caption[Mission 3 fuel cost]{Fuel cost comparison for Mission~3. All aircraft complete the mission; costs reflect aggregate fleet consumption where $n > 1$.}
  \label{tab:m3-fuel-cost}
  \begin{tabular}{lcrc}
    \toprule
    Aircraft & $n$ & Total Fuel Cost & \$/klb-nm \\
    \midrule
    DC-8* & 1 & \$36,527 & \$0.61 \\
    GV & 6 & \$115,218 & \$1.92 \\
    P-8 & 2 & \$61,273 & \$1.02 \\
    \textbf{767-200ER} & \textbf{1} & \textbf{\$78,907} & \textbf{\$1.32} \\
    A330-200 & 1 & \$136,549 & \$2.28 \\
    777-200LR & 1 & \$105,688 & \$1.76 \\
    \bottomrule
  \end{tabular}

  \smallskip
  {\footnotesize *DC-8 cost unreliable (approximately 40\% underburn).}
\end{table}

Mission~3 costs are higher per klb-nm than Missions~1--2 because low-altitude flight at slow speed is inherently less fuel-efficient than optimized high-altitude cruise. The 767 at \$1.32/klb-nm is the best single-aircraft option with reliable numbers. The P-8 fleet at \$1.02/klb-nm is competitive and demonstrates the P-8's strength on shorter, less demanding missions.

\subsection{Implications}

Mission~3 does not discriminate among candidates on feasibility---all pass. The discrimination is on cost:

\begin{itemize}
  \item The \textbf{767} has the best reliable single-aircraft cost efficiency.
  \item The \textbf{P-8 fleet} (2 aircraft) is cheaper per unit of science delivered, though at the cost of operating two platforms.
  \item The \textbf{A330} is the most expensive option (\$2.28/klb-nm) due to its high OEW driving fuel consumption even with mission-sized loading.
  \item The \textbf{777} benefits substantially from mission-sized loading but remains more expensive than the 767.
\end{itemize}

The key takeaway for Mission~3 is that aircraft size does not determine cost efficiency in a simple way. The mission-sized fuel loading equalizes the playing field, and the dominant cost driver becomes the aircraft's OEW-to-payload ratio and aerodynamic efficiency at off-design conditions.
