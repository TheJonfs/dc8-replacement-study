\subsection{Calibration Procedure}

Each aircraft model is calibrated by adjusting four parameters to minimize the root-mean-square error between predicted and published ranges at three corner points of the range-payload diagram:

\begin{enumerate}
    \item \textbf{Maximum payload} with fuel to MTOW
    \item \textbf{Maximum fuel} with payload to MTOW
    \item \textbf{Maximum fuel} with zero payload (ferry)
\end{enumerate}

The four calibration parameters are:

\begin{itemize}
    \item $C_{D_0}$---zero-lift drag coefficient
    \item $e$---Oswald efficiency factor
    \item $k_{\text{adj}}$---TSFC adjustment multiplier
    \item $f_{oh}$---non-cruise fuel overhead fraction
\end{itemize}

Optimization uses differential evolution (a global optimizer) followed by Nelder-Mead refinement. The P-8 is not calibrated independently; it is derived from the calibrated 737-900ER model with modifications to OEW, fuel capacity, and Oswald efficiency reflecting the raked wingtip installation.

\subsection{Results}

\begin{table}[htbp]
    \centering
    \small
    \caption[Calibration results]{Calibration results for all study aircraft. The P-8 is derived from the 737-900ER; convergence refers to the parent model.}
    \label{tab:calibration-results}
    \begin{tabular}{@{}lccccccc@{}}
        \toprule
        Aircraft & $C_{D_0}$ & $e$ & $k_{\text{adj}}$ & $f_{oh}$ & $L/D$ max & RMS Error & Converged \\
        \midrule
        DC-8-72    & 0.0141 & 0.968 & 0.605 & 0.260 & 20.4 & 0.5\%  & Yes \\
        G-V        & 0.0150 & 0.745 & 0.801 & 0.131 & 17.3 & 1.2\%  & Yes \\
        737-900ER  & 0.0355 & 1.223 & 0.757 & 0.062 & 16.7 & 0.7\%  & Yes \\
        P-8        & 0.0597 & 0.975 & 0.339 & 0.180 & 11.5 & 0.0\%  & Yes$^*$ \\
        767-200ER  & 0.0177 & 0.732 & 0.951 & 0.030 & 16.1 & 0.0\%  & Yes \\
        A330-200   & 0.0334 & 2.165 & 1.021 & 0.000 & 22.6 & 3.0\%  & Yes \\
        777-200LR  & 0.0410 & 1.811 & 0.556 & 0.080 & 18.3 & 6.5\%  & No \\
        \bottomrule
        \multicolumn{8}{@{}l}{\footnotesize $^*$Derived from 737-900ER; convergence refers to the parent model.}
    \end{tabular}
\end{table}

\subsection{Confidence Assessment}

The calibrated parameters fall into three confidence tiers:

\subsubsection{High Confidence: Boeing 767-200ER}

All four parameters are physically reasonable. The zero-lift drag coefficient ($C_{D_0} = 0.0177$) is consistent with a clean widebody transport. The Oswald efficiency ($e = 0.732$) is within the expected range for a moderate aspect ratio wing. The TSFC adjustment ($k_{\text{adj}} = 0.951$) is close to unity, meaning the published TSFC closely predicts actual fuel consumption. The overhead fraction ($f_{oh} = 0.030$) is small, consistent with efficient climb and descent profiles. The RMS range error is effectively zero ($2 \times 10^{-12}$, i.e., machine precision). Mission results for the 767-200ER can be treated as quantitatively reliable.

\subsubsection{Medium Confidence: Gulfstream G-V}

Parameters are generally physical. The $C_{D_0} = 0.0150$ is appropriate for a clean business jet. The Oswald efficiency ($e = 0.745$) is reasonable. The TSFC adjustment ($k_{\text{adj}} = 0.801$) indicates the model needs approximately 20\% less fuel consumption than the published TSFC suggests, which is somewhat low but not unphysical---it may absorb inaccuracies in the published TSFC value or reflect the G-V's efficient flight profile. The overhead fraction ($f_{oh} = 0.131$) is higher than expected but plausible for a business jet with steep climb profiles. The RMS error is 1.2\%, acceptable for this analysis.

\subsubsection{Low Confidence: DC-8, P-8, A330-200, 777-200LR}

These four aircraft exhibit at least one unphysical calibration parameter:

\textbf{DC-8-72}: The TSFC adjustment $k_{\text{adj}} = 0.605$ implies the model needs only 60\% of the published fuel consumption rate. This produces fuel burn rates approximately 40\% below reality. The likely cause is compensating interaction between $k_{\text{adj}}$ and the high overhead fraction $f_{oh} = 0.260$: the model allocates 26\% of takeoff weight to non-cruise overhead and then uses an artificially low burn rate for the remaining cruise fuel. The individual parameters are not physically interpretable, though their combined effect reproduces the range-payload data accurately (RMS 0.5\%).

\textbf{P-8 Poseidon}: The zero-lift drag coefficient $C_{D_0} = 0.0597$ is 3--4 times the expected value for a 737-derived airframe. This artifact propagates from the 737-900ER parent calibration ($C_{D_0} = 0.0355$, already high) and is amplified by the P-8 derivation process. The practical consequence is that engine-out performance cannot be reliably modeled: at $C_{D_0} = 0.06$, drag exceeds available thrust at all cruise altitudes when an engine fails.

\textbf{A330-200}: The Oswald efficiency $e = 2.165$ is unphysical---values above 1.0 violate the theoretical upper bound for a planar wing. The overhead fraction $f_{oh} \approx 0$ is implausibly low. These artifacts likely arise from compensating errors in the calibration, possibly related to inaccuracies in the published A330-200 range data for the specific MTOW variant modeled. The 3.0\% RMS error is the second-highest among all aircraft.

\textbf{777-200LR}: The Oswald efficiency $e = 1.811$ is unphysical. The TSFC adjustment $k_{\text{adj}} = 0.556$ implies only 56\% of published fuel consumption. The $C_{D_0} = 0.041$ is approximately double the expected value. The calibration did not converge (RMS 6.5\%, highest of all aircraft), suggesting the four-parameter model is insufficient to capture the 777-200LR's performance characteristics from the available range-payload data.

\subsection{Range-Payload Diagrams}

Range-payload diagrams for all six study aircraft are presented in \cref{fig:rp-overlay} (overlay) and \cref{fig:rp-dc8,fig:rp-gv,fig:rp-p8,fig:rp-767,fig:rp-a330,fig:rp-777} (individual). Each diagram shows the three calibration corner points (markers) and the model-predicted range-payload curve. Good agreement between the markers and the curve indicates successful calibration; deviation (as seen for the A330-200 and 777-200LR) indicates the model does not fully capture the aircraft's performance characteristics.

\begin{figure}[htbp]
    \centering
    \includegraphics[width=\textwidth]{figures/rp_overlay_all.png}
    \caption[Range-payload overlay for all aircraft]{Range-payload overlay for all six study aircraft, showing calibration corner points and model-predicted curves.}
    \label{fig:rp-overlay}
\end{figure}

\begin{figure}[htbp]
    \centering
    \includegraphics[width=0.8\textwidth]{figures/rp_dc8.png}
    \caption[DC-8-72 range-payload diagram]{Range-payload diagram for the DC-8-72 (NASA N817NA).}
    \label{fig:rp-dc8}
\end{figure}

\begin{figure}[htbp]
    \centering
    \includegraphics[width=0.8\textwidth]{figures/rp_gv.png}
    \caption[G-V range-payload diagram]{Range-payload diagram for the Gulfstream G-V.}
    \label{fig:rp-gv}
\end{figure}

\begin{figure}[htbp]
    \centering
    \includegraphics[width=0.8\textwidth]{figures/rp_p8.png}
    \caption[P-8 range-payload diagram]{Range-payload diagram for the Boeing P-8 Poseidon.}
    \label{fig:rp-p8}
\end{figure}

\begin{figure}[htbp]
    \centering
    \includegraphics[width=0.8\textwidth]{figures/rp_767200er.png}
    \caption[767-200ER range-payload diagram]{Range-payload diagram for the Boeing 767-200ER.}
    \label{fig:rp-767}
\end{figure}

\begin{figure}[htbp]
    \centering
    \includegraphics[width=0.8\textwidth]{figures/rp_a330200.png}
    \caption[A330-200 range-payload diagram]{Range-payload diagram for the Airbus A330-200.}
    \label{fig:rp-a330}
\end{figure}

\begin{figure}[htbp]
    \centering
    \includegraphics[width=0.8\textwidth]{figures/rp_777200lr.png}
    \caption[777-200LR range-payload diagram]{Range-payload diagram for the Boeing 777-200LR.}
    \label{fig:rp-777}
\end{figure}

\subsection{Implications for Mission Analysis}

The calibration quality directly determines the confidence level of mission results. The practical implication is:

\begin{itemize}
    \item \textbf{767-200ER}: Absolute fuel burn numbers, costs, and range predictions are trustworthy.
    \item \textbf{G-V}: Results are directionally reliable; absolute numbers carry moderate uncertainty.
    \item \textbf{DC-8, P-8, A330-200, 777-200LR}: Useful for relative comparison and feasibility assessment (pass/fail), but absolute fuel consumption and cost numbers carry substantial uncertainty. The DC-8's fuel burn is systematically underestimated by approximately 40\%.
\end{itemize}

This tiered confidence structure is carried through \cref{sec:mission1,sec:mission2,sec:mission3}, where results are presented with explicit confidence ratings.
