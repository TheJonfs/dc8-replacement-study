% Conclusions and Recommendations
% Input file for main.tex — \section{} header is defined there

\subsection{Principal Findings}
\label{subsec:principal-findings}

This study evaluated five candidate aircraft against the DC-8-72 Flying Laboratory across three representative airborne science missions. The analysis yields the following principal findings:

\noindent\textbf{1. The Boeing 767-200ER is the strongest single-aircraft replacement candidate.}

The 767 is the only aircraft that passes all three missions with high model confidence. It demonstrates:
\begin{itemize}
    \item Mission~1 (engine-out): PASS with 25,732~lb of fuel remaining at destination, maintaining cruise altitude above 33,000~ft throughout the engine-out segment.
    \item Mission~2 (vertical sampling): PASS with 18 climb-descend cycles, progressive ceiling from 41,000 to 43,100~ft.
    \item Mission~3 (low-altitude endurance): PASS with reliable fuel consumption (9,865~lb/hr average).
    \item Cost efficiency of \$0.57--\$1.32/klb-nm across missions, consistently the best among aircraft with reliable calibrations.
\end{itemize}

\noindent\textbf{2. The Boeing P-8 Poseidon is the strongest fleet-based alternative.}

The P-8 cannot complete Mission~1 but offers competitive performance on Missions~2 and~3:
\begin{itemize}
    \item Mission~2: 3-aircraft fleet at \$0.83/klb-nm with progressive ceiling (38,000 to 40,000~ft).
    \item Mission~3: 2-aircraft fleet at \$1.02/klb-nm, the cheapest fleet option.
    \item Its 737-derived narrow-body fuselage closely matches the DC-8's width.
    \item Existing military production maintains parts supply and supports structural modification.
\end{itemize}

\noindent\textbf{3. The DC-8 cannot perform the most demanding missions, confirming the need for replacement.}

The DC-8 fails Mission~1 by 1,863~nm, demonstrating that long-range engine-out missions exceed its capability. While the DC-8 passes Missions~2 and~3, its age-related operational challenges (crew shortages, spare parts, ground handling) compound the performance limitation.

\noindent\textbf{4. Calibration quality limits the quantitative conclusions for four of six aircraft.}

Only the 767-200ER and G-V produce calibration parameters within physical bounds. The DC-8, P-8, A330-200, and 777-200LR exhibit at least one unphysical parameter, making their absolute fuel consumption and cost numbers unreliable. The qualitative feasibility conclusions (pass/fail) are more robust than the quantitative cost metrics.

\noindent\textbf{5. Mission-specific modeling innovations are critical for fair comparison.}

Three modeling choices significantly affected results:
\begin{itemize}
    \item \textbf{Engine-out modeling} (Mission~1): Properly models the altitude and fuel consumption penalty of losing an engine, discriminating between aircraft that can and cannot complete the mission.
    \item \textbf{Progressive ceiling} (Mission~2): The two-regime thrust lapse model enables physically realistic ceiling increases as fuel burns off, capturing the scientists' top priority of flying higher.
    \item \textbf{Mission-sized fuel loading} (Mission~3): Iterative fuel sizing prevents penalizing aircraft for fuel capacity they don't need, reducing the 777's cost metric by 60\%.
\end{itemize}

\subsection{Recommendations}
\label{subsec:recommendations}

\noindent\textbf{Primary Recommendation: Boeing 767-200ER.}

The 767-200ER is recommended as the primary replacement candidate based on:
\begin{itemize}
    \item \textbf{Performance}: Only aircraft passing all missions with high confidence
    \item \textbf{Cost efficiency}: Best reliable \$/klb-nm on every mission
    \item \textbf{Fuselage}: Semi-wide-body with adequate length for spatially separated instruments
    \item \textbf{Configuration}: Conventional low-wing, underwing-engine, conventional-tail layout (no T-tail obstruction)
    \item \textbf{Structural amenability}: Boeing conventional construction supports cutouts and external mounts
    \item \textbf{Supply chain}: KC-46 tanker production maintains Boeing 767 parts availability
    \item \textbf{Crew availability}: Large 767 pilot population with established training infrastructure
\end{itemize}

\noindent\textbf{Secondary Recommendation: Boeing P-8 Poseidon (Fleet Role).}

The P-8 merits consideration as a complement to a 767 primary platform or as a focused-mission aircraft for campaigns not requiring Mission~1-class range:
\begin{itemize}
    \item Competitive fleet costs on shorter missions
    \item Narrow-body fuselage matches DC-8 instrument configuration
    \item Military production line provides modified airframes and structural provisions
    \item 2--3 aircraft fleet is operationally feasible (unlike G-V's 6--9 aircraft)
\end{itemize}

\noindent\textbf{Not Recommended as Primary Replacement.}

\begin{itemize}
    \item \textbf{Gulfstream G-V}: Excellent altitude capability (47,000~ft) but fleet sizes of 6--9 aircraft are impractical for routine campaign operations.
    \item \textbf{Airbus A330-200}: Highest Mission~3 cost, uncertain Mission~1 status, and low model confidence. The A330's high OEW penalizes it on every mission.
    \item \textbf{Boeing 777-200LR}: Likely capable but the most expensive single-aircraft option by a substantial margin. Its capability far exceeds the mission requirements, suggesting it is overspecified for the science mission set.
\end{itemize}

\subsection{Caveats and Future Work}
\label{subsec:caveats-future-work}

This study is limited to fuel-based performance and cost comparisons using publicly available data. A complete replacement evaluation should also consider:

\begin{itemize}
    \item \textbf{Acquisition cost and availability} of retired or in-service airframes
    \item \textbf{Modification cost} for laboratory conversion, sensor installations, and structural reinforcements
    \item \textbf{Certification requirements} for research operations, including experimental and supplemental type certificates
    \item \textbf{Operating cost beyond fuel}, including crew, maintenance, insurance, and airport fees
    \item \textbf{Fleet transition logistics}, including timeline, crew transition training, and parallel operations during the transition period
\end{itemize}

The calibration quality limitations identified in this study could be addressed through:

\begin{itemize}
    \item \textbf{Higher-fidelity performance models} (e.g., NASA's Flight Optimization System or equivalent)
    \item \textbf{Direct collaboration with manufacturers} to obtain more detailed performance data
    \item \textbf{Flight test data} from candidate aircraft operating at the specific conditions studied here
\end{itemize}
