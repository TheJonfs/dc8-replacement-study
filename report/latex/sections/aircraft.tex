Six aircraft are modeled and compared: the DC-8-72 baseline and five candidates spanning a range of sizes from a business jet to a long-range widebody. This section summarizes the key specifications and the rationale for each candidate's inclusion.

\subsection{Douglas DC-8-72 (NASA N817NA)---Baseline}

The DC-8 Flying Laboratory is the benchmark against which all candidates are measured. The aircraft in its NASA-modified configuration has the following characteristics:

\begin{table}[htbp]
    \centering
    \caption[DC-8-72 specifications]{DC-8-72 (NASA N817NA) key specifications.}
    \label{tab:dc8-specs}
    \begin{tabular}{@{}ll@{}}
        \toprule
        Parameter & Value \\
        \midrule
        MTOW & 325,000 lb \\
        OEW & 157,000 lb \\
        Max Payload & 52,000 lb \\
        Max Fuel & 147,255 lb \\
        Engines & 4 $\times$ CFM56-2-C1 \\
        Cruise Mach & 0.80 \\
        Service Ceiling & 42,000 ft \\
        Wing Area & 2,868 ft\textsuperscript{2} \\
        Aspect Ratio & 7.68 \\
        \bottomrule
    \end{tabular}
\end{table}

The DC-8 features the longest commercially manufactured narrow-body fuselage, which scientists exploit for spatially separated simultaneous measurements. Its T-tail configuration, however, obstructs upper-hemisphere sensor views and interferes with solar flux instruments. The four-engine layout provides thrust redundancy valued by some investigators. The aircraft rarely reaches its service ceiling until late in a flight when fuel burn has reduced weight sufficiently.

\subsection{Gulfstream G-V}

The G-V represents the small end of the candidate range---a high-performance business jet with excellent altitude capability but limited payload capacity.

\begin{table}[htbp]
    \centering
    \caption[G-V specifications]{Gulfstream G-V key specifications.}
    \label{tab:gv-specs}
    \begin{tabular}{@{}ll@{}}
        \toprule
        Parameter & Value \\
        \midrule
        MTOW & 90,500 lb \\
        OEW & 48,200 lb \\
        Max Payload & 5,800 lb \\
        Max Fuel & 41,300 lb \\
        Engines & 2 $\times$ Rolls-Royce BR710-A1-10 (14,750 lbf each) \\
        Cruise Mach & 0.80 \\
        Service Ceiling & 51,000 ft \\
        Wing Area & 1,137 ft\textsuperscript{2} \\
        Aspect Ratio & 7.69 \\
        \bottomrule
    \end{tabular}
\end{table}

The G-V's 51,000~ft ceiling directly addresses the scientists' top priority of flying higher. However, its 5,800~lb maximum payload means that missions requiring DC-8-class payloads (30,000--52,000~lb) would require fleets of 6--9 aircraft operating simultaneously. The G-V already serves ASP in a smaller-scale research capacity.

\subsection{Boeing P-8 Poseidon}

The P-8 is a military maritime patrol aircraft derived from the Boeing 737-900ER. Its detailed performance specifications are not publicly available; the model is constructed by calibrating a 737-900ER baseline and then applying known P-8 modifications.

\begin{table}[htbp]
    \centering
    \caption[P-8 specifications]{Boeing P-8 Poseidon key specifications.}
    \label{tab:p8-specs}
    \begin{tabular}{@{}ll@{}}
        \toprule
        Parameter & Value \\
        \midrule
        MTOW (ramp) & 188,200 lb \\
        OEW & 90,995 lb \\
        Max Payload & 23,885 lb \\
        Max Fuel & 73,320 lb \\
        Engines & 2 $\times$ CFM56-7B27 (27,300 lbf each) \\
        Cruise Mach & 0.785 \\
        Service Ceiling & 41,000 ft \\
        Wing Area & 1,344 ft\textsuperscript{2} \\
        Aspect Ratio & 10.26 \\
        \bottomrule
    \end{tabular}
\end{table}

Key modifications from the 737-900ER baseline include raked wingtips (reducing induced drag), removal of approximately 7,500~lb of passenger furnishings, and auxiliary fuel tanks increasing fuel capacity from 46,063~lb to 73,320~lb. The P-8's appeal lies in its existing military infrastructure, availability of structurally modified airframes, and moderate size suitable for focused science campaigns.

\subsection{Boeing 767-200ER}

The 767-200ER occupies the middle of the size range---large enough to carry full DC-8-class payloads on a single aircraft, yet not so large as to impose excessive operating costs.

\begin{table}[htbp]
    \centering
    \caption[767-200ER specifications]{Boeing 767-200ER key specifications.}
    \label{tab:767-specs}
    \begin{tabular}{@{}ll@{}}
        \toprule
        Parameter & Value \\
        \midrule
        MTOW & 395,000 lb \\
        OEW & 179,080 lb \\
        Max Payload & 80,920 lb \\
        Max Fuel & 162,000 lb \\
        Engines & 2 $\times$ GE CF6-80C2B2 (52,500 lbf each) \\
        Cruise Mach & 0.80 \\
        Service Ceiling & 43,100 ft \\
        Wing Area & 3,050 ft\textsuperscript{2} \\
        Aspect Ratio & 7.99 \\
        \bottomrule
    \end{tabular}
\end{table}

The 767's fuselage, while wider than the DC-8's, retains a semi-narrow-body character (interior width approximately 15~ft) that does not waste floor space in the manner scientists feared with military cargo aircraft. Its 80,920~lb maximum payload exceeds the DC-8's 52,000~lb capacity, and its 162,000~lb fuel capacity provides substantial range margin. The 767 has a conventional low-wing, underwing-engine, conventional-tail layout amenable to structural modification.

\subsection{Airbus A330-200}

The A330-200 is a twin-aisle widebody offering very high fuel capacity and payload capability.

\begin{table}[htbp]
    \centering
    \caption[A330-200 specifications]{Airbus A330-200 key specifications.}
    \label{tab:a330-specs}
    \begin{tabular}{@{}ll@{}}
        \toprule
        Parameter & Value \\
        \midrule
        MTOW & 533,519 lb \\
        OEW & 265,900 lb \\
        Max Payload & 108,908 lb \\
        Max Fuel & 245,264 lb \\
        Engines & 2 $\times$ GE CF6-80E1A4 (72,000 lbf each) \\
        Cruise Mach & 0.82 \\
        Service Ceiling & 41,100 ft \\
        Wing Area & 3,892 ft\textsuperscript{2} \\
        Aspect Ratio & 10.06 \\
        \bottomrule
    \end{tabular}
\end{table}

The A330 provides the highest payload capacity among the candidates, which could support larger campaign configurations with more instruments and investigators. Its wider fuselage (interior width approximately 17~ft) offers flexibility for payload arrangement but may be wider than optimal for the wall-mounted instrument racks that scientists prefer. The A330's high OEW (265,900~lb) means it carries substantial structural weight even on lighter missions.

\subsection{Boeing 777-200LR}

The 777-200LR is the largest and longest-range candidate, designed for ultra-long-haul commercial service.

\begin{table}[htbp]
    \centering
    \caption[777-200LR specifications]{Boeing 777-200LR key specifications.}
    \label{tab:777-specs}
    \begin{tabular}{@{}ll@{}}
        \toprule
        Parameter & Value \\
        \midrule
        MTOW & 766,000 lb \\
        OEW & 320,000 lb \\
        Max Payload & 135,000 lb \\
        Max Fuel & 325,300 lb \\
        Engines & 2 $\times$ GE90-110B1 (110,100 lbf each) \\
        Cruise Mach & 0.84 \\
        Service Ceiling & 43,100 ft \\
        Wing Area & 4,702 ft\textsuperscript{2} \\
        Aspect Ratio & 9.61 \\
        \bottomrule
    \end{tabular}
\end{table}

The 777-200LR has the highest MTOW, fuel capacity, and range of any candidate, with a manufacturer-quoted maximum range exceeding 9,000~nm. Its massive GE90-110B1 engines produce more than twice the thrust of any other candidate's powerplants. While this capability provides enormous operational flexibility, the aircraft's size imposes proportionally high fuel consumption even when the mission does not require its full capability. The 777's wide fuselage (interior width approximately 19~ft) is the widest among the candidates.

\subsection{Specification Summary}

\begin{table}[htbp]
    \centering
    \small
    \caption[Aircraft specification summary]{Summary of key specifications for all candidate aircraft.}
    \label{tab:spec-summary}
    \begin{tabular}{@{}lrrrrllr@{}}
        \toprule
        Aircraft & MTOW & OEW & Max Payload & Max Fuel & Engines & Mach & Ceiling \\
                 & (lb) & (lb) & (lb) & (lb) & & & (ft) \\
        \midrule
        DC-8-72    & 325,000 & 157,000 & 52,000  & 147,255 & 4$\times$CFM56-2-C1  & 0.80  & 42,000 \\
        G-V        & 90,500  & 48,200  & 5,800   & 41,300  & 2$\times$BR710       & 0.80  & 51,000 \\
        P-8        & 188,200 & 90,995  & 23,885  & 73,320  & 2$\times$CFM56-7B27  & 0.785 & 41,000 \\
        767-200ER  & 395,000 & 179,080 & 80,920  & 162,000 & 2$\times$CF6-80C2B2  & 0.80  & 43,100 \\
        A330-200   & 533,519 & 265,900 & 108,908 & 245,264 & 2$\times$CF6-80E1A4  & 0.82  & 41,100 \\
        777-200LR  & 766,000 & 320,000 & 135,000 & 325,300 & 2$\times$GE90-110B1  & 0.84  & 43,100 \\
        \bottomrule
    \end{tabular}
\end{table}
