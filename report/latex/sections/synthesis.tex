% Cross-Mission Synthesis
% Input file for main.tex — \section{} header is defined there

\subsection{Feasibility Summary}
\label{subsec:feasibility-summary}

\begin{table}[htbp]
    \centering
    \caption[Mission feasibility summary]{Mission feasibility summary for all candidate aircraft.}
    \label{tab:feasibility-summary}
    \begin{tabular}{lccc}
        \toprule
        \textbf{Aircraft} & \textbf{Mission 1} & \textbf{Mission 2} & \textbf{Mission 3} \\
        \midrule
        DC-8-72        & \textit{FAIL} & PASS               & PASS \\
        GV             & \textit{FAIL} & PASS               & PASS \\
        P-8            & \textit{FAIL} & PASS               & PASS \\
        \textbf{767-200ER} & \textbf{PASS} & \textbf{PASS}  & \textbf{PASS} \\
        A330-200       & \textit{UNCERTAIN} & PASS (marginal) & PASS \\
        777-200LR      & LIKELY PASS   & PASS               & PASS \\
        \bottomrule
    \end{tabular}
\end{table}

The 767-200ER is the only aircraft that passes all three missions with high model confidence. The 777-200LR likely passes all three but its model confidence is low. The DC-8, GV, and P-8 all fail Mission~1, confirming that the demanding engine-out transport scenario is the most discriminating test.

\subsection{Fuel Cost Efficiency Comparison}
\label{subsec:fuel-cost-efficiency}

\Cref{fig:fuel-cost-comparison} shows the fuel cost comparison across all three missions.

\begin{figure}[htbp]
    \centering
    \includegraphics[width=0.9\textwidth]{figures/fuel_cost_comparison.png}
    \caption[Fuel cost comparison across missions]{Fuel cost comparison across all three missions for each candidate aircraft. Costs are expressed in dollars per thousand pounds of payload per nautical mile (\$/klb-nm).}
    \label{fig:fuel-cost-comparison}
\end{figure}

\subsubsection{Consolidated Cost Table (\$/klb-nm)}
\label{subsubsec:consolidated-cost}

\begin{table}[htbp]
    \centering
    \caption[Consolidated fuel cost metrics]{Consolidated fuel cost metrics (\$/klb-nm) for all candidate aircraft across three missions. Fleet size $n$ indicates the number of aircraft required.}
    \label{tab:consolidated-cost}
    \begin{tabular}{lclclcl}
        \toprule
        \textbf{Aircraft} & $n$ & \textbf{Mission 1} & $n$ & \textbf{Mission 2} & $n$ & \textbf{Mission 3} \\
        \midrule
        DC-8         & 1 & \textit{FAIL}   & 1 & \$0.44* & 1 & \$0.61* \\
        GV           & 8 & \textit{FAIL}   & 9 & \$1.24  & 6 & \$1.92 \\
        P-8          & 2 & \textit{FAIL}   & 3 & \$0.83  & 2 & \$1.02 \\
        \textbf{767-200ER} & \textbf{1} & \textbf{\$0.57} & \textbf{1} & \textbf{\$0.61} & \textbf{1} & \textbf{\$1.32} \\
        A330-200     & 1 & \textit{UNCERTAIN} & 1 & \$0.81  & 1 & \$2.28 \\
        777-200LR    & 1 & \$1.15**        & 1 & \$1.22  & 1 & \$1.76 \\
        \bottomrule
        \multicolumn{7}{l}{\footnotesize *DC-8 costs unreliable ($k_{\text{adj}} = 0.605$, approximately 40\% underburn).} \\
        \multicolumn{7}{l}{\footnotesize **LIKELY PASS; low confidence.} \\
    \end{tabular}
\end{table}

The 767 is consistently the cheapest single-aircraft option with reliable numbers: \$0.57--\$1.32/klb-nm across all missions. Its cost range spans a factor of 2.3$\times$ from the most efficient mission (Mission~1) to the least efficient (Mission~3), reflecting the transition from near-optimal cruise to severely off-design low-altitude conditions.

The P-8 fleet offers competitive costs on Missions~2 and~3 (\$0.83 and \$1.02/klb-nm respectively) but cannot address Mission~1. The GV fleet is the most expensive option on every mission it can complete.

\subsubsection{Cost Drivers}
\label{subsubsec:cost-drivers}

Three factors drive the cost differences:

\begin{enumerate}
    \item \textbf{OEW-to-payload ratio}: The weight the aircraft carries for its own structure relative to the science payload. The 767 carries 179,080~lb of structure for up to 80,920~lb of payload (ratio 2.2:1). The 777 carries 320,000~lb for up to 135,000~lb (2.4:1). The A330 carries 265,900~lb for up to 108,908~lb (2.4:1). Lower ratios mean less fuel consumed hauling structure.

    \item \textbf{Aerodynamic efficiency}: The aircraft's $L/D$ ratio at mission conditions. At cruise altitude, the 767 achieves $L/D$ of approximately 16.1. At low altitude (Mission~3), $L/D$ drops to 15.6 for the 767 but only 7.4 for the P-8 (driven by its unphysical $C_{D_0}$).

    \item \textbf{Engine efficiency}: The TSFC at operating conditions. The 767's CF6-80C2B2 engines with $k_{\text{adj}} = 0.951$ indicate that the published TSFC is a good predictor of actual consumption. Aircraft with low $k_{\text{adj}}$ values (DC-8 at 0.605, P-8 at 0.339) have artificially low fuel consumption in the model.
\end{enumerate}

\subsection{Fleet vs.\ Single-Aircraft Operations}
\label{subsec:fleet-vs-single}

The GV and P-8 require multi-aircraft fleets for missions with payloads exceeding their individual capacity:

\begin{table}[htbp]
    \centering
    \caption[Fleet sizing requirements]{Fleet sizing requirements for the GV and P-8 compared to the 767 single-aircraft solution.}
    \label{tab:fleet-sizing}
    \begin{tabular}{lccc}
        \toprule
        \textbf{Mission} & \textbf{GV Fleet} & \textbf{P-8 Fleet} & \textbf{767 Single} \\
        \midrule
        Mission 1 (46,000 lb) & 8 aircraft (\textit{FAIL}) & 2 aircraft (\textit{FAIL}) & 1 aircraft (PASS) \\
        Mission 2 (52,000 lb) & 9 aircraft & 3 aircraft & 1 aircraft \\
        Mission 3 (30,000 lb) & 6 aircraft & 2 aircraft & 1 aircraft \\
        \bottomrule
    \end{tabular}
\end{table}

Fleet operations impose costs beyond fuel:
\begin{itemize}
    \item \textbf{Coordination complexity}: Multiple aircraft must be scheduled, maintained, and crewed simultaneously.
    \item \textbf{Aggregate weight penalty}: OEW is multiplied by fleet size. The GV 9$\times$ fleet aggregate OEW (433,800~lb) exceeds the 777's single-aircraft OEW (320,000~lb).
    \item \textbf{Scientific coordination}: Measurements from multiple aircraft must be temporally and spatially coordinated, adding data processing complexity.
    \item \textbf{Infrastructure}: Each aircraft requires ground handling, maintenance, and crew accommodations.
\end{itemize}

The P-8 fleet (2--3 aircraft) is more operationally feasible than the GV fleet (6--9 aircraft), but both impose substantially more logistical overhead than a single-aircraft solution.

\subsection{Aircraft Ranking}
\label{subsec:aircraft-ranking}

Based on the quantitative analysis, the candidates rank as follows:

\begin{enumerate}
    \item \textbf{Boeing 767-200ER}: Only aircraft passing all missions with high confidence. Best single-aircraft cost efficiency on reliable numbers. Adequate altitude capability (43,100~ft peak).

    \item \textbf{Boeing P-8 Poseidon}: Competitive fleet-based costs on Missions~2--3. Fails Mission~1. Fleet size (2--3 aircraft) is operationally feasible.

    \item \textbf{Boeing 777-200LR}: Likely passes all missions but at approximately double the 767's cost. Massive capability exceeds mission requirements. Low model confidence.

    \item \textbf{Airbus A330-200}: High OEW drives highest Mission~3 cost (\$2.28/klb-nm). Uncertain Mission~1 status. Low model confidence.

    \item \textbf{Gulfstream G-V}: Highest altitude capability (47,000~ft) but fleet sizes of 6--9 aircraft are operationally impractical for routine campaign use.

    \item \textbf{Douglas DC-8-72}: Fails Mission~1. Cost numbers unreliable. Confirms replacement need.
\end{enumerate}
