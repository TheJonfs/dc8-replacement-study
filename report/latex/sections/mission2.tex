\subsection{Mission Definition}

\begin{table}[htbp]
  \centering
  \caption[Mission 2 parameters]{Mission 2 definition parameters.}
  \label{tab:m2-definition}
  \begin{tabular}{ll}
    \toprule
    Parameter & Value \\
    \midrule
    Route & Christchurch, New Zealand (NZCH) to Punta Arenas, Chile (SCCI) \\
    Distance & 4,200 nm \\
    Payload & 52,000 lb \\
    Profile & Repeating climb-descend cycles (5,000 ft to ceiling) \\
    Fuel budget & Explicit reserves (5\% + 200 nm alternate + 30 min hold) \\
    \bottomrule
  \end{tabular}
\end{table}

This mission exercises the aircraft's ability to perform vertical atmospheric column sampling---a core DC-8 science capability. The aircraft repeatedly climbs from 5,000~ft to its thrust-limited ceiling and descends back, with each cycle sampling the full atmospheric column. As fuel burns off and the aircraft lightens, it becomes capable of reaching progressively higher altitudes on successive cycles.

This profile directly addresses the scientists' top priority: ``flying higher is the single most desirable capability improvement.''

\subsection{Results}

\subsubsection{Feasibility Assessment}

All six aircraft pass Mission~2:

\begin{table}[htbp]
  \centering
  \caption[Mission 2 feasibility]{Mission 2 feasibility assessment. All aircraft complete the mission; discrimination is on ceiling altitude and progressive capability.}
  \label{tab:m2-feasibility}
  \small
  \begin{tabular}{lcccccc}
    \toprule
    Aircraft & Confidence & $n$ & Cycles & Init.\ Ceiling (ft) & Peak Ceiling (ft) & Progression \\
    \midrule
    DC-8 & Low & 1 & 21 & 42,000 & 42,000 & Flat \\
    GV & Medium & 9 & 16 & 44,000 & 47,000 & +3,000 ft \\
    P-8 & Low & 3 & 14 & 38,000 & 40,000 & +2,000 ft \\
    \textbf{767-200ER} & \textbf{High} & 1 & 18 & 41,000 & 43,100 & \textbf{+2,100 ft} \\
    A330-200 & Low & 1 & 19 & 41,100 & 41,100 & Flat \\
    777-200LR & Low & 1 & 17 & 43,100 & 43,100 & Flat \\
    \bottomrule
  \end{tabular}
\end{table}

Mission~2 is less discriminating than Mission~1---all aircraft can complete it---but the quality of the science output varies significantly.

\subsubsection{Progressive Ceiling Analysis}

The progressive ceiling chart (\cref{fig:profile-m2-ceiling}) is the most scientifically significant visualization in this study. It shows how each aircraft's achievable ceiling evolves across successive climb-descend cycles.

\begin{figure}[htbp]
  \centering
  \includegraphics[width=0.85\textwidth]{figures/profile_m2_ceiling.png}
  \caption[Mission 2 progressive ceiling]{Progressive ceiling altitude across successive climb-descend cycles for Mission~2. Aircraft that lighten appreciably during the mission reach higher altitudes on later cycles, sampling more of the atmospheric column.}
  \label{fig:profile-m2-ceiling}
\end{figure}

Three aircraft exhibit meaningful progressive ceiling increase:

\begin{itemize}
  \item \textbf{GV}: From 44,000 to 47,000~ft (+3,000~ft over 16 cycles). The GV reaches the highest altitude of any candidate, but requires a fleet of 9 aircraft.
  \item \textbf{767-200ER}: From 41,000 to 43,100~ft (+2,100~ft over 18 cycles). The 767 demonstrates progressive capability gain with high-confidence results.
  \item \textbf{P-8}: From 38,000 to 40,000~ft (+2,000~ft over 14 cycles). The P-8's ceiling is artificially depressed by its unphysical $C_{D_0}$; the real P-8 ceiling of 41,000~ft would likely show higher values.
\end{itemize}

Three aircraft show flat ceilings:

\begin{itemize}
  \item \textbf{DC-8}: Flat at 42,000~ft across all 21 cycles. This is likely a calibration artifact: the $k_{\text{adj}} = 0.605$ produces approximately 40\% underburn, so the aircraft does not get light enough fast enough for the ceiling to rise within the mission.
  \item \textbf{777-200LR}: Flat at 43,100~ft across 17 cycles. The 777 has so much thrust that its service ceiling (a structural/pressurization limit, not thrust-limited) governs from cycle~1.
  \item \textbf{A330-200}: Flat at 41,100~ft across 19 cycles. The unphysical $e = 2.165$ makes the drag model unreliable at altitude.
\end{itemize}

\subsubsection{Altitude Profile}

The sawtooth altitude profile (\cref{fig:profile-m2-altitude}) shows the raw climb-descend pattern for all six aircraft overlaid.

\begin{figure}[htbp]
  \centering
  \includegraphics[width=0.85\textwidth]{figures/profile_m2_altitude.png}
  \caption[Mission 2 altitude profile]{Sawtooth altitude profile for Mission~2. Each aircraft repeatedly climbs from 5,000~ft to its thrust-limited ceiling and descends. Six overlaid patterns create a visually dense plot; the progressive ceiling chart (\cref{fig:profile-m2-ceiling}) extracts the key metric more clearly.}
  \label{fig:profile-m2-altitude}
\end{figure}

The profile is visually dense because six overlaid sawtooth patterns create many intersections. The ceiling progression chart (\cref{fig:profile-m2-ceiling}) extracts the key metric more clearly and should be preferred for decision-making.

\subsubsection{Weight Breakdown}

\cref{fig:weight-breakdown-m2} shows the weight breakdown for Mission~2.

\begin{figure}[htbp]
  \centering
  \includegraphics[width=0.85\textwidth]{figures/weight_breakdown_m2.png}
  \caption[Mission 2 weight breakdown]{Weight breakdown comparison for Mission~2 (vertical atmospheric sampling). Fleet aggregates are shown for the GV~($9\times$) and P-8~($3\times$).}
  \label{fig:weight-breakdown-m2}
\end{figure}

The GV $9\times$ fleet aggregate (814,000~lb) exceeds any single aircraft, illustrating the resource penalty of fleet operations. The P-8 $3\times$ fleet (545,000~lb) is comparable to the A330 (534,000~lb). The 767 (393,000~lb) is the lightest single-aircraft solution capable of carrying the full 52,000~lb payload.

\subsection{Fuel Cost}

\begin{table}[htbp]
  \centering
  \caption[Mission 2 fuel cost]{Fuel cost comparison for Mission~2. All aircraft complete the mission; costs reflect aggregate fleet consumption where $n > 1$.}
  \label{tab:m2-fuel-cost}
  \begin{tabular}{lcrc}
    \toprule
    Aircraft & $n$ & Total Fuel Cost & \$/klb-nm \\
    \midrule
    DC-8* & 1 & \$95,224 & \$0.44 \\
    GV & 9 & \$269,828 & \$1.24 \\
    P-8 & 3 & \$180,564 & \$0.83 \\
    \textbf{767-200ER} & \textbf{1} & \textbf{\$132,985} & \textbf{\$0.61} \\
    A330-200 & 1 & \$177,001 & \$0.81 \\
    777-200LR & 1 & \$267,037 & \$1.22 \\
    \bottomrule
  \end{tabular}

  \smallskip
  {\footnotesize *DC-8 cost unreliable due to $k_{\text{adj}} = 0.605$ (approximately 40\% underburn).}
\end{table}

The 767 offers the best cost efficiency among aircraft with reliable calibrations (\$0.61/klb-nm). The DC-8 appears cheapest (\$0.44/klb-nm) but this figure is artificially low.

\subsection{Scientific Value Assessment}

From a science mission perspective, the key question is not just ``can the aircraft complete the mission'' but ``how much useful data does it collect?'' The relevant factors are:

\begin{enumerate}
  \item \textbf{Peak altitude}: Higher ceilings sample more of the atmospheric column. The GV wins (47,000~ft), followed by the 767 and 777 (43,100~ft).
  \item \textbf{Progressive ceiling}: Aircraft that reach higher altitudes on later cycles sample the upper atmosphere when instruments have been fully characterized during lower-altitude cycles. The GV, 767, and P-8 show this behavior.
  \item \textbf{Number of cycles}: More cycles provide more vertical profiles. The DC-8 leads (21 cycles), followed by the A330 (19) and 767 (18).
  \item \textbf{Single-aircraft operation}: All payloads on one aircraft allows coordinated measurements. Only the DC-8, 767, A330, and 777 can carry the full 52,000~lb.
\end{enumerate}

The 767 provides the best combination: progressive ceiling (41,000 to 43,100~ft), adequate cycle count (18), single-aircraft payload capacity, and high-confidence model results.
